\documentclass[12pt]{article}
\pagestyle{empty}
\usepackage[utf8]{inputenc}
\usepackage[russian]{babel}
\usepackage[top=1cm,left=1cm,right=2cm, bottom=1cm]{geometry}
\usepackage{tabularx}
\begin{document}


\begin{center}
\quad Университет ИТМО, факультет программной инженерии и компьютерной техники \\
\quad Двухнедельная отчётная работа по «Информатике»: аннотация к статье\\
\quad Дата прошедшей лекции: \underline{27.09.2023} 	Номер прошедшей лекции: \underline{1}	Дата сдачи: \underline{11.10.2023}

\bigskip

\quad Выполнил(а) \underline{Миронов Иван Николаевич}, № группы \underline{ P3132 }, оценка \underline{\hspace{2cm}}


\end{center}

\begin{tabularx}{\textwidth} { 
  | >{\raggedright\arraybackslash}X|}
    \hline
\textbf{Название статьи/главы книги/видеолекции}\\
    \textit{Как развитие алгоритмов сжатия остановилось 20 лет назад, или о новом конкурсе на 200 тысяч евро}\\
    \hline
\end{tabularx}

\begin{tabularx}{\textwidth} 
{ 
| >{\centering\arraybackslash}X
| >{\centering\arraybackslash}X
| >{\centering\arraybackslash}X 
|}
    \textbf{ФИО автора статьи \quad (или e-mail)} & \textbf{Дата публикации \qquad\qquad (не старше 2020 года)} & \textbf{Размер статьи \qquad\qquad (от 400 слов)} \\
     \textit{3Dvideo} & <<\underline{02}>> \underline{августа} 2021 г. & \underline{5254 слов} \\
    \hline
\end{tabularx}

\begin{tabularx}{\textwidth} { 
  | >{\raggedright\arraybackslash}X|}
    \textbf{Прямая полная ссылка на источник или сокращённая ссылка (bit.ly, tr.im и т.п.)} 
    \textit{https://habr.com/ru/articles/570694/}
    \smallskip\\
    \hline
    \textbf{Теги, ключевые слова или словосочетания}\\
    \textit{Программирование, Сжатие данных, Машинное обучение, Научно-популярное, Искусственный интеллект}
    \smallskip\\
    \hline
    \textbf{Перечень фактов, упомянутых в статье (минимум три пункта)}
    \begin{enumerate}
    	\item Алгоритмы сжатия данных не стоят на месте
	\item Основная идея контекстного моделирования - это предскзание следующего символа
    	\item В алгоритмах сжатия данных без потерь активно применяются нейросетевые подходы
    	\item Алгоритмы сжатия специфичных данных без потерь часто лучше универсальных 
    	
	\end{enumerate}
    \\ \hline
    \textbf{Позитивные следствия и/или достоинства описанной в статье технологии (минимум три пункта)}
    \begin{enumerate}
	\item Развитие алгоритмов сжатия не стоит на месте
    	\item Нейросетеывые подходы поозволили сфере не стоять на месте
    	\item Сейчас проводятся соревнования, которые привлекают специалистов к области сжатия данных без потерь
    \end{enumerate}
    \\ \hline
    \textbf{Негативные следствия и/или достоинства описанной в статье технологии (минимум три пункта)}
    \begin{enumerate}
    	\item Нейросетеывые подходы сжатия информации требуют больших вычислительных ресурсов
    	\item Анализируя количество выпускаемых статей, можно выяснить, что в области сжатия данных работает мало специалистов
    	\item Осведомленность людей о новых алгоритмах сжатия данных без потерь очень мала
	\item Современные нейросетевые подходы сжатия данных работают на GPU
    \end{enumerate}
    \\ \hline
    \textbf{Ваши замечания, пожелания преподавателю или анекдот о программистах}\footnote{Наличие этой графы не влияет на оценку}\\
    \bigskip\
    \bigskip\\
    \hline
    
\end{tabularx}


\end{document}
