\documentclass[12pt]{article}
\pagestyle{empty}
\usepackage[utf8]{inputenc}
\usepackage[russian]{babel}
\usepackage[top=1cm,left=1cm,right=2cm, bottom=1cm]{geometry}
\usepackage{tabularx}
\usepackage{graphicx}
\graphicspath{ {.} }
\begin{document}


\begin{center}
\quad Университет ИТМО, факультет программной инженерии и компьютерной техники \\
\quad Двухнедельная отчётная работа по «Информатике»: аннотация к статье\\
\quad Дата прошедшей лекции: \underline{13.11.2023} 	Номер прошедшей лекции: \underline{7}	Дата сдачи: \underline{20.12.2023}

\bigskip

\quad Выполнил \underline{Миронов Иван Николаевич}, № группы \underline{ P3132 }, оценка \underline{\hspace{2cm}}


\end{center}

\begin{tabularx}{\textwidth} { 
  | >{\raggedright\arraybackslash}X|}
    \hline
\textbf{Название статьи/главы книги/видеолекции}\\
    \textit{Чего ждать от NeoVim: особенности редактора}\\
    \hline
\end{tabularx}

\begin{tabularx}{\textwidth} 
{ 
| >{\centering\arraybackslash}X
| >{\centering\arraybackslash}X
| >{\centering\arraybackslash}X 
|}
    \textbf{ФИО автора статьи \quad (или e-mail)} & \textbf{Дата публикации \qquad\qquad (не старше 2020 года)} & \textbf{Размер статьи \qquad\qquad (от 400 слов)} \\
     \textit{antgubarev} & <<\underline{19}>> \underline{августа} 2023 г. & \underline{2449 слов} \\
    \hline
\end{tabularx}

\begin{tabularx}{\textwidth} { 
  | >{\raggedright\arraybackslash}X|}
    \textbf{Прямая полная ссылка на источник или сокращённая ссылка (bit.ly, tr.im и т.п.)} 
    \textit{https://habr.com/ru/companies/avito/articles/682962/}
    \smallskip\\
    \hline
    \textbf{Теги, ключевые слова или словосочетания}\\
    \textit{vim, редакторы кода}
    \smallskip\\
    \hline
    \textbf{Перечень фактов, упомянутых в статье}
    \begin{enumerate}
	\item В neovim нет привычных вкладок, как в обычных редакторах кода. Вместо вкладок в neovim буфферы, окна и табы
	\item Управление в neovim происходит только с помощью клавиатуры
	\item В neovim есть удобные команды для работы с программой или просто текстом, которые можно комбинировать
	\end{enumerate}
    \\ \hline
    \textbf{Позитивные следствия и/или достоинства описанной в статье технологии}
    \begin{enumerate}
    	\item Кастомизация, с помощью которой можно настроить каждую деталь 
	\item Существуют тысячи готовых сборок neovim со всеми нужными плагинами
	\item Есть возможность самому запрограммировать плагин для neovim на Lua
    \end{enumerate}
    \\ \hline
    \textbf{Негативные следствия и/или достоинства описанной в статье технологии}
    \begin{enumerate}
	\item Сложно привыкнуть к различным комбинациям клавиш
	\item В neovim'е "из коробки"\ нет практически ничего
	\item Нет графического интерфейса
    \end{enumerate}
    \\ \hline
    \textbf{Ваши замечания, пожелания преподавателю или анекдот о программистах}\footnote{Наличие этой графы не влияет на оценку}\\

	 \raisebox{-\totalheight}{\includegraphics[width=6cm]{meme_7_ann.jpg}}\\
	 \hline

\end{tabularx}


\end{document}
