\documentclass[12pt]{article}
\pagestyle{empty}
\usepackage[utf8]{inputenc}
\usepackage[russian]{babel}
\usepackage[top=1cm,left=1cm,right=2cm, bottom=1cm]{geometry}
\usepackage{tabularx}
\usepackage{graphicx}
\graphicspath{ {.} }
\begin{document}


\begin{center}
\quad Университет ИТМО, факультет программной инженерии и компьютерной техники \\
\quad Двухнедельная отчётная работа по «Информатике»: аннотация к статье\\
\quad Дата прошедшей лекции: \underline{25.10.2023} 	Номер прошедшей лекции: \underline{4}	Дата сдачи: \underline{09.11.2023}

\bigskip

\quad Выполнил(а) \underline{Миронов Иван Николаевич}, № группы \underline{ P3132 }, оценка \underline{\hspace{2cm}}


\end{center}

\begin{tabularx}{\textwidth} { 
  | >{\raggedright\arraybackslash}X|}
    \hline
\textbf{Название статьи/главы книги/видеолекции}\\
    \textit{MarkedText — маркдаун здорового человека}\\
    \hline
\end{tabularx}

\begin{tabularx}{\textwidth} 
{ 
| >{\centering\arraybackslash}X
| >{\centering\arraybackslash}X
| >{\centering\arraybackslash}X 
|}
    \textbf{ФИО автора статьи \quad (или e-mail)} & \textbf{Дата публикации \qquad\qquad (не старше 2020 года)} & \textbf{Размер статьи \qquad\qquad (от 400 слов)} \\
     \textit{nin-jin} & <<\underline{8}>> \underline{января} 2021 г. & \underline{2458 слов} \\
    \hline
\end{tabularx}

\begin{tabularx}{\textwidth} { 
  | >{\raggedright\arraybackslash}X|}
    \textbf{Прямая полная ссылка на источник или сокращённая ссылка (bit.ly, tr.im и т.п.)} 
    \textit{https://habr.com/ru/articles/536448/}
    \smallskip\\
    \hline
    \textbf{Теги, ключевые слова или словосочетания}\\
    \textit{markown, языки разметки}
    \smallskip\\
    \hline
    \textbf{Перечень фактов, упомянутых в статье}
    \begin{enumerate}
	\item Некоторые метасимволы Markdown есть только в английской раскладке. 
	\item В Markdown нет удобного инструментария для редактирования таблиц.
	\item Markdown не единственный легковесный язык разметки.
	\end{enumerate}
    \\ \hline
    \textbf{Позитивные следствия и/или достоинства описанной в статье технологии}
    \begin{enumerate}
    	\item MarkedText использует уинверсальные для всех языков метасимволы.
    	\item Метасимволы MarkedText логичнее, чем метасимволы Markdown.
	\item Таблицы в MarkedText сохраняют двумерность, но при этом позволяют не выравнивать ячейки вручную.
    \end{enumerate}
    \\ \hline
    \textbf{Негативные следствия и/или достоинства описанной в статье технологии}
    \begin{enumerate}
	\item Поддерживает только локальный редактор.
    	\item Сложно привыкнуть к новому синтаксису.
	\item Спорная реализация ссылок. Ссылки в Markdown реализованы лаконичнее.
    \end{enumerate}
    \\ \hline
    \textbf{Ваши замечания, пожелания преподавателю или анекдот о программистах}\footnote{Наличие этой графы не влияет на оценку}\\
    \includegraphics[width=3.5cm]{meme_4_ann.jpg}

    
\end{tabularx}


\end{document}
