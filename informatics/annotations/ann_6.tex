\documentclass[12pt]{article}
\pagestyle{empty}
\usepackage[utf8]{inputenc}
\usepackage[russian]{babel}
\usepackage[top=1cm,left=1cm,right=2cm, bottom=1cm]{geometry}
\usepackage{tabularx}
\usepackage{graphicx}
\graphicspath{ {.} }
\begin{document}


\begin{center}
\quad Университет ИТМО, факультет программной инженерии и компьютерной техники \\
\quad Двухнедельная отчётная работа по «Информатике»: аннотация к статье\\
\quad Дата прошедшей лекции: \underline{22.11.2023} 	Номер прошедшей лекции: \underline{6}	Дата сдачи: \underline{6.12.2023}

\bigskip

\quad Выполнил(а) \underline{Миронов Иван Николаевич}, № группы \underline{ P3132 }, оценка \underline{\hspace{2cm}}


\end{center}

\begin{tabularx}{\textwidth} { 
  | >{\raggedright\arraybackslash}X|}
    \hline
\textbf{Название статьи/главы книги/видеолекции}\\
    \textit{Typst — современная альтернатива LaTeX}\\
    \hline
\end{tabularx}

\begin{tabularx}{\textwidth} 
{ 
| >{\centering\arraybackslash}X
| >{\centering\arraybackslash}X
| >{\centering\arraybackslash}X 
|}
    \textbf{ФИО автора статьи \quad (или e-mail)} & \textbf{Дата публикации \qquad\qquad (не старше 2020 года)} & \textbf{Размер статьи \qquad\qquad (от 400 слов)} \\
     \textit{Albert\_Wesker} & <<\underline{20}>> \underline{июня} 2023 г. & \underline{1745 слов} \\
    \hline
\end{tabularx}

\begin{tabularx}{\textwidth} { 
  | >{\raggedright\arraybackslash}X|}
    \textbf{Прямая полная ссылка на источник или сокращённая ссылка (bit.ly, tr.im и т.п.)} 
    \textit{https://habr.com/ru/companies/timeweb/articles/742756/}
    \smallskip\\
    \hline
    \textbf{Теги, ключевые слова или словосочетания}\\
    \textit{офисные пакеты, отечественное ПО, импортозамещение}
    \smallskip\\
    \hline
    \textbf{Перечень фактов, упомянутых в статье}
    \begin{enumerate}
		\item Мотивации создания \LaTeX\ и Typst очень похожи, разработчиков Typst'a разочаровал \LaTeX\ и они решили разрабоать свой язык разметки текста
		\item На данный простые языки разметки на подобие Markdown очень популярны
		\item Именно \LaTeX\ привнес в \TeX\ идею описательной разметки
	\end{enumerate}
    \\ \hline
    \textbf{Позитивные следствия и/или достоинства описанной в статье технологии}
    \begin{enumerate}
    	\item Typst имеет упрощенный синтаксис по сравнению с \LaTeX\ 
		\item В Typst есть гибкая система стилей похожая на CSS
		\item Скорость работы Typst превосходит скорость \LaTeX\, потому что Typst напсан на Rust
		\item Typst имеет гораздо лучший пользовательский интерфейс с удобной отладкой, чего иногда не скажешь о \LaTeX\
		\item Есть удобный онлайн редактор от разработчиков самого языка разметки
    \end{enumerate}
    \\ \hline
    \textbf{Негативные следствия и/или достоинства описанной в статье технологии}
    \begin{enumerate}
		\item Из-за простого синтаксиса Typst нельзя выполнить технически сложную верстку
	    \item Сложный переход от всем привычного \LaTeX\ к Typst
		\item Маленькое комьюнити и соответсвенно малое количество ответов на вопросы на форумах
    \end{enumerate}
    \\ \hline
    \textbf{Ваши замечания, пожелания преподавателю или анекдот о программистах}\footnote{Наличие этой графы не влияет на оценку}\\

	 \raisebox{-\totalheight}{\includegraphics[width=5.5cm]{meme_6_ann.jpg}}\\
	 \hline

\end{tabularx}


\end{document}
