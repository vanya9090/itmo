\documentclass[12pt]{article}
\pagestyle{empty}
\usepackage[utf8]{inputenc}
\usepackage[russian]{babel}
\usepackage[top=1cm,left=1cm,right=2cm, bottom=1cm]{geometry}
\usepackage{tabularx}
\begin{document}


\begin{center}
\quad Университет ИТМО, факультет программной инженерии и компьютерной техники \\
\quad Двухнедельная отчётная работа по «Информатике»: аннотация к статье\\
\quad Дата прошедшей лекции: \underline{13.09.2023} 	Номер прошедшей лекции: \underline{1}	Дата сдачи: \underline{27.09.2023}

\bigskip

\quad Выполнил(а) \underline{Миронов Иван Николаевич}, № группы \underline{ P3132 }, оценка \underline{\hspace{2cm}}


\end{center}

\begin{tabularx}{\textwidth} { 
  | >{\raggedright\arraybackslash}X|}
    \hline
\textbf{Название статьи/главы книги/видеолекции}\\
    \textit{Приложения алгебры кортежей. Часть 1. Гибкая система счисления с простыми основаниями}\\
    \hline
\end{tabularx}

\begin{tabularx}{\textwidth} 
{ 
| >{\centering\arraybackslash}X
| >{\centering\arraybackslash}X
| >{\centering\arraybackslash}X 
|}
    \textbf{ФИО автора статьи \quad (или e-mail)} & \textbf{Дата публикации \qquad\qquad (не старше 2020 года)} & \textbf{Размер статьи \qquad\qquad (от 400 слов)} \\
     \textit{AntiLogik} & <<\underline{07}>> \underline{августа} 2023 г. & \underline{3095 слов} \\
    \hline
\end{tabularx}

\begin{tabularx}{\textwidth} { 
  | >{\raggedright\arraybackslash}X|}
    \textbf{Прямая полная ссылка на источник или сокращённая ссылка (bit.ly, tr.im и т.п.)} 
    \textit{https://habr.com/ru/articles/752836/}
    \smallskip\\
    \hline
    \textbf{Теги, ключевые слова или словосочетания}\\
    \textit{Высокая производительность, Алгоритмы, Математика, Сжатие данных}
    \smallskip\\
    \hline
    \textbf{Перечень фактов, упомянутых в статье (минимум три пункта)}
    \begin{enumerate}
    	\item FPR - это система счисления, в которой каждое число предствалено кортежем из атрибутов и компонент
    	\item Каждый атрибут представляет из себя упорядоченную последовательность простых чисел, а каждая компонента - последовательность целых чисел
    	\item С помощью фиктивных компонент и атрибутов можно определить операции умножения и деления
    	
	\end{enumerate}
    \\ \hline
    \textbf{Позитивные следствия и/или достоинства описанной в статье технологии (минимум три пункта)}
    \begin{enumerate}
    	\item  Система счисления позволяет выразить все рациональные числа, некоторые иррациональные и мнимые числа
    	\item Предложенная система счисления позволяет экономить место в памяти при хранении больших чисел
    	\item Произведение чисел не требует больших вычислительных ресурсов, а именно O(n), где n - количество атрибутов
    \end{enumerate}
    \\ \hline
    \textbf{Негативные следствия и/или достоинства описанной в статье технологии (минимум три пункта)}
    \begin{enumerate}
    	\item Нет общего алгоритма для построения суммы чисел
    	\item Нельзя задать 0
    	\item Перевод в позиционные системы счисления требует больших вычислительных ресурсов
    \end{enumerate}
    \\ \hline
    \textbf{Ваши замечания, пожелания преподавателю или анекдот о программистах}\footnote{Наличие этой графы не влияет на оценку}\\
    \bigskip\\
    \bigskip\\
    \hline
    
\end{tabularx}


\end{document}
