\documentclass[12pt]{article}
\pagestyle{empty}
\usepackage[utf8]{inputenc}
\usepackage[russian]{babel}
\usepackage[top=1cm,left=1cm,right=2cm, bottom=1cm]{geometry}
\usepackage{tabularx}
\usepackage{graphicx}
\graphicspath{ {.} }
\begin{document}


\begin{center}
\quad Университет ИТМО, факультет программной инженерии и компьютерной техники \\
\quad Двухнедельная отчётная работа по «Информатике»: аннотация к статье\\
\quad Дата прошедшей лекции: \underline{09.11.2023} 	Номер прошедшей лекции: \underline{5}	Дата сдачи: \underline{22.11.2023}

\bigskip

\quad Выполнил(а) \underline{Миронов Иван Николаевич}, № группы \underline{ P3132 }, оценка \underline{\hspace{2cm}}


\end{center}

\begin{tabularx}{\textwidth} { 
  | >{\raggedright\arraybackslash}X|}
    \hline
\textbf{Название статьи/главы книги/видеолекции}\\
    \textit{Краш, крипота или кринж? Тестирую 7 антисанкционных офисных пакетов — альтернатив Microsoft Office}\\
    \hline
\end{tabularx}

\begin{tabularx}{\textwidth} 
{ 
| >{\centering\arraybackslash}X
| >{\centering\arraybackslash}X
| >{\centering\arraybackslash}X 
|}
    \textbf{ФИО автора статьи \quad (или e-mail)} & \textbf{Дата публикации \qquad\qquad (не старше 2020 года)} & \textbf{Размер статьи \qquad\qquad (от 400 слов)} \\
     \textit{Yconilev} & <<\underline{8}>> \underline{декабря} 2022 г. & \underline{5698 слов} \\
    \hline
\end{tabularx}

\begin{tabularx}{\textwidth} { 
  | >{\raggedright\arraybackslash}X|}
    \textbf{Прямая полная ссылка на источник или сокращённая ссылка (bit.ly, tr.im и т.п.)} 
    \textit{https://habr.com/ru/articles/704000/}
    \smallskip\\
    \hline
    \textbf{Теги, ключевые слова или словосочетания}\\
    \textit{офисные пакеты, отечественное ПО, импортозамещение}
    \smallskip\\
    \hline
    \textbf{Перечень фактов, упомянутых в статье}
    \begin{enumerate}
	\item Существует большая проблема отечественного офисного ПО - недобросовестные компании "переклеивают" ярлыки импортного ПО и выдают за свое.  
	\item Сейчас в России проблема импортозамещения офисного ПО стоит на высоком уровне.
	\item Проблема импортозамещения офисного ПО в России очень актуальна.
	\item Объем рынка офисного ПО 2021 года составляет около 50 млрд. рублей.
	\end{enumerate}
    \\ \hline
    \textbf{Позитивные следствия и/или достоинства описанной в статье технологии}
    \begin{enumerate}
    	\item Некоторые отечественные офисные пакеты превосходят LibreOffice по функциональности, например в некоторых из них есть возможность совместной realtime работы
	\item Отечественные офисные пакеты совметимы с российскими ОС 
	\item Офисные пакеты "МойОфис" и "Р7-Офис" практическим ничем не уступают LibreOffice, а в некоторых вещах даже превосходят его
    \end{enumerate}
    \\ \hline
    \textbf{Негативные следствия и/или достоинства описанной в статье технологии}
    \begin{enumerate}
	\item Офисное ПО компании AlterOffice является всего лишь клоном LibreOffice с переделанным интерфейсом.
    	\item ПО "Офис+" можно запустить только на Windows, на Astra Linux он работать не будет.
	\item ПО "Р7-Офис" неправильно работает с автоматическом заполнением ячеек и макросами.
    \end{enumerate}
    \\ \hline
    \textbf{Ваши замечания, пожелания преподавателю или анекдот о программистах}\footnote{Наличие этой графы не влияет на оценку}\\
\bigskip \\
\hline

\end{tabularx}


\end{document}
