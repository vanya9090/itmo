\documentclass[12pt]{article}
\usepackage[utf8]{inputenc}
\usepackage[a4paper, top=15mm, right=10mm, bottom=15mm, left=5mm, headsep=-5mm]{geometry}
\usepackage[english,russian]{babel}
\usepackage{tikz}
\usepackage{fancyhdr}
\usepackage{lettrine} 
\usepackage{GoudyIn}
\usepackage{mathtools}
\usepackage{amssymb}
%\usepackage{graphicx, type1cm, lettrine}

\usetikzlibrary{calc}


\fancyheadoffset[R]{5cm}
\fancyhead[C]{26\qquad КНИГА I ПРЕДЛ. II. ЗАДАЧА\\}
\renewcommand{\footrulewidth}{0 mm}
\renewcommand{\headrulewidth}{0 mm}
\renewcommand\LettrineFontHook{\GoudyInfamily}
\fancyfoot{}


\begin{document}
\thispagestyle{fancy}
\begin{minipage}[H]{0.3\textwidth}
\begin{tikzpicture}[line width=2pt]
\coordinate (D) at (0, 0);
\coordinate (B) at (7.5mm, -7.5mm);
\coordinate (E) at ((22mm, -20mm);
\coordinate (C) at (-13mm,-7mm);
\coordinate (F) at (30mm,0mm);
\coordinate (A) at (10mm, 0mm)

% \draw (B) node [left]  {\scriptsize B} -- (D) node [below right]  {\scriptsize D};
\draw[draw=yellow] (E) node [below right]  {\scriptsize E} -- (B) node [right]  {\scriptsize B};
\draw[draw=black] (C) node [below right]  {\scriptsize C} -- (B) node [right]  {\scriptsize B};
\draw[dashed, draw=black] (A) node [below right]  {\scriptsize A} -- (B) node [right]  {\scriptsize B};
\draw[draw=red] (D) node [above]  {\scriptsize D} -- (A) node [below right]  {\scriptsize A};
\draw[draw=blue] (A) node -- (F) node [right]  {\scriptsize F};
\draw[draw=red] (0, 0) circle (30mm);
\draw[draw=blue] (7mm, -7mm) circle (20mm);
\draw[dashed, draw=red] (B) -- (D) node [above] {\scriptsize D};
\end{tikzpicture}
\end{minipage}
\qquad\qquad
\begin{minipage}[bs]{0.55\textwidth}
\vspace{2cm}
\lettrine[lines=4]{O}{} \textit{т данной точки\tikz\draw[line width=2pt] (0,0.2cm) -- node [above] {\scriptsize A} (1.5cm,0.2cm);отлосжить прямую равную прямой\tikz\draw[line width=2pt] (0,0.2cm) node [above] {\scriptsize B} -- (1.5cm,0.2cm) node [above]  {\scriptsize C};.}\\

\begin{center}
Проведем \tikz\draw[dashed, line width=2pt] (0,2mm) node [above] {\scriptsize A} -- (15mm,2mm) node [above]  {\scriptsize B}; (Пост. I),\\
построим  \tikz\draw[line width=2pt, draw=red, align=center] (0,17mm) node [above] {\scriptsize A} -- (15mm,15mm) node [above]  {\scriptsize B} -- (10mm, 5mm) node [below right] {\scriptsize C} -- (0, 17mm); (пр I.i)\\
продлим \tikz\draw[dashed, line width=2pt, draw=red] (0,2mm) node [above] {\scriptsize A} -- (15mm,2mm) node [above]  {\scriptsize B}; (пр.II)\\
опишем \tikz\draw[draw=blue, line width=2pt] (0, 0) circle (6mm) (0, 0) node [above] {\scriptsize A} -- (-6mm, 0mm) node [left] {\scriptsize C}; (пост. III), и \tikz\draw[draw=red, line width=2pt] (0, 0) circle (11mm) (0, 0) node [above] {\scriptsize D} -- (8mm, -8mm) node [below right] {\scriptsize E}; (пост. III)\\
продлим  \tikz\draw[draw=red, line width=2pt] (0,2mm) node [above] {\scriptsize D} -- (15mm,2mm) node [above]  {\scriptsize A}; (пост. II),\\
тогда искомая пряма это \tikz\draw[draw=blue, line width=2pt] (0,2mm) node [above] {\scriptsize A} -- (15mm,2mm) node [above]  {\scriptsize F};.\\
\bigskip\\
Поскольку \tikz\draw[draw=yellow, line width=2pt] (0,2mm) node [above] {\scriptsize E} -- (15mm,2mm) node [above]  {\scriptsize D}; = \tikz\draw[draw=blue, line width=2pt] (0,2mm) node [above] {\scriptsize D} -- (15mm,2mm) node [above]  {\scriptsize F}; (опр. 15),\\
и \tikz\draw[dashed, draw=red, line width=2pt] (0,2mm) node [above] {\scriptsize B} -- (15mm,2mm) node [above]  {\scriptsize D}; = \tikz\draw[draw=red, line width=2pt] (0,2mm) node [above] {\scriptsize D} -- (15mm,2mm) node [above]  {\scriptsize A}; (постр.),\\
$\therefore$ \tikz\draw[draw=yellow, line width=2pt] (0,2mm) node [above] {\scriptsize B} -- (15mm,2mm) node [above]  {\scriptsize E}; = \tikz\draw[draw=blue, line width=2pt] (0,2mm) node [above] {\scriptsize A} -- (15mm,2mm) node [above]  {\scriptsize A}; (акс. III),\\
но (опр.15) \tikz\draw[draw=black, line width=2pt] (0,2mm) node [above] {\scriptsize B} -- (15mm,2mm) node [above]  {\scriptsize C}; = \tikz\draw[draw=red, line width=2pt] (0,2mm) node [above] {\scriptsize B} -- (15mm,2mm) node [above]  {\scriptsize E}; = \tikz\draw[draw=red, line width=2pt] (0,2mm) node [above] {\scriptsize A} -- (15mm,2mm) node [above]  {\scriptsize F};.\\
\bigskip\\
$\therefore$ \tikz\draw[draw=blue, line width=2pt] (0,2mm) node [above] {\scriptsize A} -- (15mm,2mm) node [above]  {\scriptsize F}; проведенная из данной точки\\
(\tikz\draw[draw=red, line width=2pt] (0,2mm) -- node [above] {\scriptsize A} (15mm,2mm);) равна данной прямой \tikz\draw[draw=black, line width=2pt] (0,2mm) node [above] {\scriptsize B} -- (15mm,2mm) node [above]  {\scriptsize C}; (акс. I)


\begin{flushright}
ч. т. д.
\end{flushright}
\end{center}
\end{minipage}
\end{document}
